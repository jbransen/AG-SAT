
An \emph{Attribute Grammar} is a context-free grammar with inherited
and synthesized attributes assigned to every non-terminal
and a graph that represents dependencies between those attributes
at the production level.

\begin{definition}{Attribute Grammar}

An Attribute Grammar (AG) is a triple $\langle G,A,D\rangle$, where:
\begin{itemize}
 \item Context-free grammar $G = \langle N,T,P,S\rangle$ contains a set of
    non-terminals $N$, a set of terminals $T$, a set of production
    rules $P$ and a start symbol $S$. Every $p\in P$ is of the form
    $X_{p,0}\rightarrow X_{p,1}\ldots X_{p,\mid p\mid}$, with $lhs(p) = X_{p,0}$
    and $rhs(p) = X_{p,1},\ldots ,X_{p,\mid p\mid}$, where each $X_{p,i}\in
    \{lhs(p)\}\cup rhs(p)$ is called a field of $p$ and an occurrence of
    some non-terminal $X\in N$, i.e. $\exis{X\in N}{X_{p,i} = X}$.
 \item A set $A(X) = A_{inh}(X) \cup A_{syn}(X)$ is defined for all $X\in N$.
    From the attributes we infer the attribute occurrences gathered in 
    the set $A_P(p) = A_{in}(p) \cup A_{out}(p)$, where $A_{in}(p)$ 
    and $A_{out}(p)$ are the input and output occurrences of $p$ respectively.
    \begin{equation*}
     \begin{aligned}
        A_{in}(p) = &\set{X_{p,0} \cdot a}{X \cdot a \in A_{inh}(lhs(p))}\\
        \cup &\set{X_{p,i} \cdot a}{X_{p,i} \in rhs(p) 
                                \sand X = X_{p,i}
                                \sand X \cdot a \in A_{syn}(X)) }
     \end{aligned}
    \end{equation*}
    \begin{equation*}
     \begin{aligned}
        A_{out}(p) = &\set{X_{p,0} \cdot a}{X \cdot a \in A_{syn}(lhs(p))}\\
        \cup &\set{X_{p,i} \cdot a}{X_{p,i} \in rhs(p) 
                                \sand X = X_{p,i}
                                \sand X \cdot a \in A_{inh}(X)) }
     \end{aligned}
    \end{equation*}
 \item A dependency graph $D(p)$ indicates that attribute $a$ is used in
    the semantic function definition of attribute $b$ when 
    $(a\rightarrow b)\in D(p)$.
\end{itemize}
\end{definition}

The definition of AGs given above is not a complete definition in the sense that
it does not contain enough information to generate executable code from it -
the actual semantic function definitions are missing, for example.
However it contains all the information we need to define the problem of 
finding a static evaluation order for all Linear Ordered Attribute Grammars and
deciding whether an AG is an LOAG.

\begin{definition}{Linear Ordered Attribute Grammars}

A Linear Ordered Attribute Grammar (LOAG) is an AG that satisfies
the following properties:
\begin{itemize}
 \item For all $X \in N$ there exists a graph $R_X(x)$ that satisfies:
    \begin{itemize}
        \item \emph{Totality}: There must be an edge between every inherited
                and synthesized pair:
                \begin{equation*}
                 \begin{aligned}
                    \forall &(i \in A_{inh}(X), s\in A_{syn}(X))\\
                        &(i\rightarrow s) \in R_X(X) \lor (s\rightarrow i)
                    \in R_X(X)
                 \end{aligned}
                \end{equation*}
    \end{itemize}
 \item For all $p\in P$ there exists a graph $R_P(p)$ that satisfies:
    \begin{itemize}
        \item \emph{Feasibility}: The graph must include the 
            dependencies, i.e. 
            \begin{equation*}
                D(p) \subseteq R_P(p).
            \end{equation*}
        \item \emph{Consistency}: The graph must be consistent with $R_X$, i.e.
            \begin{equation*}
             \begin{aligned}
                \forall &(X\in \{lhs(p)\}\cup rhs(p), X = X_{p,i})\\
                    (&(X\cdot a\rightarrow X\cdot b)\in R_X(X) 
                    \Rightarrow (X_{p,i}\cdot a\rightarrow X_{p,i}\cdot b))
             \end{aligned}
            \end{equation*}
        \item \emph{Orderability}: The graph $R_P(p)$ must be acyclic.
    \end{itemize}
\end{itemize}
\end{definition}

Graph $R_P$ serves the same purpose as graph $ED_P$ from Kastens and from $R_X(X)$
we can infer Kastens' interfaces\cite{kastens80}.

\subsection{Satisfiability}

Using the above definition for LOAGs we can define a Boolean Satisfiability 
Problem that determines whether an arbitrary AG is an LOAG.

Firstly, be defining a variable $x_{i,s}$ for every element of the Cartesian
product $i \times s$, with $i \in A_{inh}(X)$ and $s \in A_{syn}(X)$ for all
$X\in N$. Secondly, by saying $x_{a,b} =\top$ when $a = (X_{p,i}\cdot a')$,
$b = (X_{p,i}\cdot b')$ and $(X_{p,i}\cdot a'\rightarrow X_{p,i}\cdot b')\in D_P(p)$
(\emph{feasibility}).
Thirdly, by assigning every variable $\top$ or $\bot$ (\emph{totality}) under 
the constraints that the graph $R_P$ the assignments imply is cycle free
(\emph{orderability}).
Note that \emph{consistency} is guaranteed by making sure that a variable
represents the edge between attributes (non-terminal level) as well as the edges 
between all the occurrences of that edge (production level).

%\begin{definition}{Boolean linear orderedness}
%\label{def:booleanlorderedness}
%Define a variable $x_{i,s}$ for every pair in $A_inh(X) \times A_syn(X)$ for all $X\in N$.
%An AG is a member of the class LOAG if we there is an assignment $x_{i,s} = \top$ 
%or $x_{i,s} = \bottom$ for every $x_{i,s}$, such that 
%\end{definition}

