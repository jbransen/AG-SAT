\documentclass{llncs}

\usepackage{amsmath}
\usepackage{amssymb}
\usepackage{tikz}

% syntax highlighting
\newcommand{\mvar}[1]{$\;\texttt{\textcolor{black}{#1}}\;$}
\newcommand{\mkey}[1]{$\;\texttt{\textcolor{orange}{#1}}\;$}
\newcommand{\mmod}[1]{$\;\texttt{\textcolor{purple}{#1}}\;$}
\newcommand{\mdef}[1]{$\;\texttt{\textcolor{blue}{#1}}\;$}
\newcommand{\mcon}[1]{$\;\texttt{\textcolor{brickred}{#1}}\;$}
\newcommand{\mtype}[1]{$\;\texttt{\textcolor{brown}{#1}}\;$}
\newcommand{\mpar}[1]{$(#1)$}
\newcommand{\mnil}[1]{\mcon{[]}}
\newcommand{\cons}[3]{\mpar{\mval{#1}\mcon{#2}\mval{#3}}}
\newcommand{\ra}[0]{$\;\rightarrow\;$}
\newcommand{\Ra}[0]{$\;\Rightarrow\;$}
\newcommand{\ind}[0]{$\;\;\;\;$}

\newcommand{\plusplus}{\mathbin{+\!\!\!+}}

%\setlength\abovedisplayskip{4pt}
%\setlength\abovecaptionskip{0pt}
%\setlength\belowdisplayskip{0pt}
%\setlength\belowcaptionskip{0pt}
%\setlength\intextsep{4pt}

% grammars
\newcommand{\nt}[1]{\texttt{#1}}
\newcommand{\pr}[1]{\textit{\texttt{#1}}}
\newcommand{\tm}[1]{\underline{#1}}
\newcommand{\lf}[1]{\texttt{#1}.}
\newcommand{\rf}[1]{@\texttt{#1}.}
\newcommand{\f}[1]{@\texttt{#1}}
\newcommand{\ff}[1]{\texttt{#1}}

\newcommand{\app}[1]{\textsc{#1}}

% sets
\newcommand{\set}[2]{\{\;#1\;\;|\;\;#2\;\}}
\newcommand{\sand}[0]{,\;}

% quantification
\newcommand{\univ}[2]{\forall(#1)\;(#2)}
\newcommand{\exis}[2]{\exists(#1)\;(#2)}

% administration
\newcommand{\toindex}[1]{\emph{#1}\index{#1}}
\newcommand{\toindexwith}[2]{\emph{#1}\index{#2}}

% visit-sequences
\newcommand{\vsrelsize}{1.5cm}
\tikzstyle{vsstyle}=[shape=rectangle,draw]

\newcommand{\myvs}[3]{
\draw[style=vsstyle] (#1 *\vsrelsize,0) rectangle (#2 *\vsrelsize, 0.9cm) node[anchor=north east] (v1#1) {\begin{varwidth}{\vsrelsize}\small{#3}\end{varwidth}};
}


% agpictures
\definecolor{light-gray}{gray}{0.83}
\newdimen\agrelsize
\setlength{\agrelsize}{0.30cm}
\tikzstyle{agnontermstyle}=[shape=circle,draw,minimum size=\agrelsize]
\tikzstyle{agchldstyle}=[shape=circle,fill=red!70,minimum size=\agrelsize]
\tikzstyle{agattrstyle}=[shape=rectangle,draw,minimum size=\agrelsize,]
\tikzstyle{agdepstyle}=[black,>=triangle 60]
\tikzstyle{agidpstyle}=[gray,>=triangle 60]
\tikzstyle{inputstyle}=[shape=rectangle,fill=light-gray,draw,minimum size=\agrelsize]
\tikzstyle{outputstyle}=[shape=rectangle,draw,minimum size=\agrelsize]
\newcommand{\agcoordpary}{6}   %% y-coordinate of parent attributes
\newcommand{\agcoordchldy}{0}  %% y-coordinate of child attributes
\newcommand{\agcoordlocy}{3}   %% y-coordinate of local attributes
\newcommand{\agbeginprod}[5]{} %% you could begin a picture here
\newcommand{\agendprod}[5]{}   %% you could end a picture here
\newcommand{\agparent}[5]{ \draw (#4 \agrelsize,\agcoordpary \agrelsize) node [style=agnontermstyle] (#3) {}; \draw (#3.north) node [anchor=south] {\footnotesize{ #2 }}; \draw (#3.south) node [anchor=north] {\footnotesize{ #5 }}; }
\newcommand{\agparattr}[6]{ \draw (#4 \agrelsize,\agcoordpary \agrelsize) node [style=#6] (#3) {}; \draw (#3.south) node [anchor=south] {\tiny{#2}}; }
\newcommand{\agchldattr}[6]{ \draw (#4 \agrelsize,\agcoordchldy \agrelsize) node [style=#6] (#3) {}; \draw (#3.north) node [anchor=north] {\tiny{#2}}; }
\newcommand{\agchldnonterm}[5]{ \draw (#4 \agrelsize,\agcoordchldy \agrelsize) node [style=agnontermstyle] (#3) {}; \draw (#3.south) node [anchor=north] {\footnotesize{ #2:#5 }}; }
\newcommand{\agchldterm}[5]{ \draw (#4 \agrelsize,\agcoordchldy \agrelsize) node [style=agchldstyle] (#3) {}; \draw (#3.south) node [anchor=north] {\footnotesize{ #2 }}; }
\newcommand{\agsibline}[3]{ \draw (#2.east) -- (#3.west); }
\newcommand{\agdeplineupdown}[4]{ \draw[->,style=agdepstyle] (#2.north) .. controls +(0,1) and +(0,-1) .. (#3.south); }
\newcommand{\agdeplineupup}[5]{ \draw[->,style=agdepstyle] (#2.north) .. controls +(0,#5) and +(0,#5) .. (#3.north); }
\newcommand{\agidplineup}[5]{ \draw[->,style=agidpstyle] (#2.north) .. controls +(0,#5) and +(0,#5) .. (#3.north); }
\newcommand{\agidplineupdown}[5]{ \draw[->,style=agidpstyle] (#2.north) .. controls +(0,#5) and +(0,-#5) .. (#3.south); }
\newcommand{\agdeplineupupbold}[4]{ \draw[->,style=agdepstyle,line width=2pt] (#2.north) .. controls +(0,1) and +(0,1) .. (#3.north); }
\newcommand{\agdeplinedowndown}[5]{ \draw[->,style=agdepstyle] (#2.south) .. controls +(0,-#5) and +(0,-#5) .. (#3.south); }
\newcommand{\agidplinedown}[5]{ \draw[->,style=agidpstyle] (#2.south) .. controls +(0,-#5) and +(0,-#5) .. (#3.south); }
\newcommand{\agdeplinedowndownbold}[4]{ \draw[->,style=agdepstyle,line width=2pt] (#2.south) .. controls +(0,-1) and +(0,-1) .. (#3.south); }
\newcommand{\agdeplinedowndowndashed}[5]{ \draw[->,style=agidpstyle, dashed] (#2.south) .. controls +(0,-#5) and +(0,-#5) .. (#3.south); }
\newcommand{\agdeplinedownup}[4]{ \draw[->,style=agdepstyle] (#2.south) .. controls +(0,-1) and +(0,1) .. (#3.north); }
\newcommand{\agpicture}[1]{ \begin{tikzpicture}\input{#1.tikz}\end{tikzpicture} }

\newcommand{\mypic}[2]{
 \begin{figure}[ht]
  \centering
    \agpicture{diagrams/#1}
    \caption{#2}
  \label{fig:#1}
 \end{figure}
}

\newcommand{\mypicstar}[2]{
 \begin{figure*}[t]
  \centering
    \agpicture{diagrams/#1}
    \caption{#2}
  \label{fig:#1}
 \end{figure*}
}

\newcommand{\mydoublepic}[6]{
 \begin{figure*}[t]
  \centering
    \begin{minipage}[a]{#3\textwidth}
        \agpicture{diagrams/#2}
    \end{minipage}
    \begin{minipage}[a]{#5\textwidth}
        \agpicture{diagrams/#4}
    \end{minipage}
    \caption{#6}
  \label{fig:#1}
 \end{figure*}
}

\newcounter{algoctr}
\setcounter{algoctr}{-1}
\newcounter{cfctr}
\setcounter{cfctr}{-1}
\newcounter{defctr}



\author{L. Thomas van Binsbergen\inst{1} \and Jeroen Bransen\inst{2} \and Koen Claessen\inst{3}}
\institute{todo, \email{todo@todo.co.uk}
\and Utrecht University, Utrecht, The Netherlands, \email{J.Bransen@uu.nl}
\and todo, \email{koen@chalmers.se}}
\title{SAT-solving Linear Ordered Attribute Grammars}

\begin{document}

\maketitle

\begin{abstract}
Attribute grammars are a formalism for describing tree computations (catamorphisms), and are therefore suitable for implementing the semantics of programming languages. A well-known subclass is the class of linear ordered attribute grammars, for which a linear evaluation order can be found statically. The problem of finding such an order is however known to to be NP-hard.

In this paper we encode linear ordered attribute grammar scheduling as a SAT-problem. We encode cycle-freeness of the dependency graphs efficiently by making the graphs chordal, which is a new approach for encoding such graph property in SAT. We show that using Minisat as an external SAT-solver the scheduling algorithm runs much faster than existing scheduling algorithms on real-world examples. Furthermore, this approach allows fine-grained control over the resulting schedule, thereby enabling new attribute grammar optimizations.

\keywords{Attribute Grammars, static analysis, SAT-solving}
\end{abstract}

\section{Introduction}

\section{Linear Ordered Attribute Grammars}

An \emph{Attribute Grammar} is a context-free grammar with inherited
and synthesized attributes assigned to every non-terminal
and a graph that represents dependencies between those attributes
at the production level.

\begin{definition}{Attribute Grammar}

An Attribute Grammar (AG) is a triple $\langle G,A,D\rangle$, where:
\begin{itemize}
 \item Context-free grammar $G = \langle N,T,P,S\rangle$ contains a set of
    non-terminals $N$, a set of terminals $T$, a set of production
    rules $P$ and a start symbol $S$. Every $p\in P$ is of the form
    $X_{p,0}\rightarrow X_{p,1}\ldots X_{p,\mid p\mid}$, with $lhs(p) = X_{p,0}$
    and $rhs(p) = X_{p,1},\ldots ,X_{p,\mid p\mid}$, where each $X_{p,i}\in
    \{lhs(p)\}\cup rhs(p)$ is called a field of $p$ and an occurrence of
    some non-terminal $X\in N$, i.e. $\exis{X\in N}{X_{p,i} = X}$.
 \item A set $A(X) = A_{inh}(X) \cup A_{syn}(X)$ is defined for all $X\in N$.
    From the attributes we infer the attribute occurrences gathered in 
    the set $A_P(p) = A_{in}(p) \cup A_{out}(p)$, where $A_{in}(p)$ 
    and $A_{out}(p)$ are the input and output occurrences of $p$ respectively.
    \begin{equation*}
     \begin{aligned}
        A_{in}(p) = &\set{X_{p,0} \cdot a}{X \cdot a \in A_{inh}(lhs(p))}\\
        \cup &\set{X_{p,i} \cdot a}{X_{p,i} \in rhs(p) 
                                \sand X = X_{p,i}
                                \sand X \cdot a \in A_{syn}(X)) }
     \end{aligned}
    \end{equation*}
    \begin{equation*}
     \begin{aligned}
        A_{out}(p) = &\set{X_{p,0} \cdot a}{X \cdot a \in A_{syn}(lhs(p))}\\
        \cup &\set{X_{p,i} \cdot a}{X_{p,i} \in rhs(p) 
                                \sand X = X_{p,i}
                                \sand X \cdot a \in A_{inh}(X)) }
     \end{aligned}
    \end{equation*}
 \item A dependency graph $D(p)$ indicates that attribute $a$ is used in
    the semantic function definition of attribute $b$ when 
    $(a\rightarrow b)\in D(p)$.
\end{itemize}
\end{definition}

The definition of AGs given above is not a complete definition in the sense that
it does not contain enough information to generate executable code from it -
the actual semantic function definitions are missing, for example.
However it contains all the information we need to define the problem of 
finding a static evaluation order for all Linear Ordered Attribute Grammars and
deciding whether an AG is an LOAG.

\begin{definition}{Linear Ordered Attribute Grammars}

A Linear Ordered Attribute Grammar (LOAG) is an AG that satisfies
the following properties:
\begin{itemize}
 \item For all $X \in N$ there exists a graph $R_X(x)$ that satisfies:
    \begin{itemize}
        \item \emph{Totality}: There must be an edge between every inherited
                and synthesized pair:
                \begin{equation*}
                 \begin{aligned}
                    \forall &(i \in A_{inh}(X), s\in A_{syn}(X))\\
                        &(i\rightarrow s) \in R_X(X) \lor (s\rightarrow i)
                    \in R_X(X)
                 \end{aligned}
                \end{equation*}
    \end{itemize}
 \item For all $p\in P$ there exists a graph $R_P(p)$ that satisfies:
    \begin{itemize}
        \item \emph{Feasibility}: The graph must include the 
            dependencies, i.e. 
            \begin{equation*}
                D(p) \subseteq R_P(p).
            \end{equation*}
        \item \emph{Consistency}: The graph must be consistent with $R_X$, i.e.
            \begin{equation*}
             \begin{aligned}
                \forall &(X\in \{lhs(p)\}\cup rhs(p), X = X_{p,i})\\
                    (&(X\cdot a\rightarrow X\cdot b)\in R_X(X) 
                    \Rightarrow (X_{p,i}\cdot a\rightarrow X_{p,i}\cdot b))
             \end{aligned}
            \end{equation*}
        \item \emph{Orderability}: The graph $R_P(p)$ must be acyclic.
    \end{itemize}
\end{itemize}
\end{definition}

Graph $R_P$ serves the same purpose as graph $ED_P$ from Kastens and from $R_X(X)$
we can infer Kastens' interfaces\cite{kastens80}.

\subsection{Satisfiability}

Using the above definition for LOAGs we can define a Boolean Satisfiability 
Problem that determines whether an arbitrary AG is an LOAG.

Firstly, be defining a variable $x_{i,s}$ for every element of the Cartesian
product $i \times s$, with $i \in A_{inh}(X)$ and $s \in A_{syn}(X)$ for all
$X\in N$. Secondly, by saying $x_{a,b} =\top$ when $a = (X_{p,i}\cdot a')$,
$b = (X_{p,i}\cdot b')$ and $(X_{p,i}\cdot a'\rightarrow X_{p,i}\cdot b')\in D_P(p)$
(\emph{feasibility}).
Thirdly, by assigning every variable $\top$ or $\bot$ (\emph{totality}) under 
the constraints that the graph $R_P$ the assignments imply is cycle free
(\emph{orderability}).
Note that \emph{consistency} is guaranteed by making sure that a variable
represents the edge between attributes (non-terminal level) as well as the edges 
between all the occurrences of that edge (production level).

%\begin{definition}{Boolean linear orderedness}
%\label{def:booleanlorderedness}
%Define a variable $x_{i,s}$ for every pair in $A_inh(X) \times A_syn(X)$ for all $X\in N$.
%An AG is a member of the class LOAG if we there is an assignment $x_{i,s} = \top$ 
%or $x_{i,s} = \bottom$ for every $x_{i,s}$, such that 
%\end{definition}



\bibliographystyle{splncs}
\bibliography{biblio}


\end{document}
