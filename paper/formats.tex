
% syntax highlighting
\newcommand{\mvar}[1]{$\;\texttt{\textcolor{black}{#1}}\;$}
\newcommand{\mkey}[1]{$\;\texttt{\textcolor{orange}{#1}}\;$}
\newcommand{\mmod}[1]{$\;\texttt{\textcolor{purple}{#1}}\;$}
\newcommand{\mdef}[1]{$\;\texttt{\textcolor{blue}{#1}}\;$}
\newcommand{\mcon}[1]{$\;\texttt{\textcolor{brickred}{#1}}\;$}
\newcommand{\mtype}[1]{$\;\texttt{\textcolor{brown}{#1}}\;$}
\newcommand{\mpar}[1]{$(#1)$}
\newcommand{\mnil}[1]{\mcon{[]}}
\newcommand{\cons}[3]{\mpar{\mval{#1}\mcon{#2}\mval{#3}}}
\newcommand{\ra}[0]{$\;\rightarrow\;$}
\newcommand{\Ra}[0]{$\;\Rightarrow\;$}
\newcommand{\ind}[0]{$\;\;\;\;$}

\newcommand{\plusplus}{\mathbin{+\!\!\!+}}

%\setlength\abovedisplayskip{4pt}
%\setlength\abovecaptionskip{0pt}
%\setlength\belowdisplayskip{0pt}
%\setlength\belowcaptionskip{0pt}
%\setlength\intextsep{4pt}

% grammars
\newcommand{\nt}[1]{\texttt{#1}}
\newcommand{\pr}[1]{\textit{\texttt{#1}}}
\newcommand{\tm}[1]{\underline{#1}}
\newcommand{\lf}[1]{\texttt{#1}.}
\newcommand{\rf}[1]{@\texttt{#1}.}
\newcommand{\f}[1]{@\texttt{#1}}
\newcommand{\ff}[1]{\texttt{#1}}

\newcommand{\app}[1]{\textsc{#1}}

% sets
\newcommand{\set}[2]{\{\;#1\;\;|\;\;#2\;\}}
\newcommand{\sand}[0]{,\;}

% quantification
\newcommand{\univ}[2]{\forall(#1)\;(#2)}
\newcommand{\exis}[2]{\exists(#1)\;(#2)}

% administration
\newcommand{\toindex}[1]{\emph{#1}\index{#1}}
\newcommand{\toindexwith}[2]{\emph{#1}\index{#2}}

% visit-sequences
\newcommand{\vsrelsize}{1.5cm}
\tikzstyle{vsstyle}=[shape=rectangle,draw]

\newcommand{\myvs}[3]{
\draw[style=vsstyle] (#1 *\vsrelsize,0) rectangle (#2 *\vsrelsize, 0.9cm) node[anchor=north east] (v1#1) {\begin{varwidth}{\vsrelsize}\small{#3}\end{varwidth}};
}


% agpictures
\definecolor{light-gray}{gray}{0.83}
\newdimen\agrelsize
\setlength{\agrelsize}{0.30cm}
\tikzstyle{agnontermstyle}=[shape=circle,draw,minimum size=\agrelsize]
\tikzstyle{agchldstyle}=[shape=circle,fill=red!70,minimum size=\agrelsize]
\tikzstyle{agattrstyle}=[shape=rectangle,draw,minimum size=\agrelsize,]
\tikzstyle{agdepstyle}=[black,>=triangle 60]
\tikzstyle{agidpstyle}=[gray,>=triangle 60]
\tikzstyle{inputstyle}=[shape=rectangle,fill=light-gray,draw,minimum size=\agrelsize]
\tikzstyle{outputstyle}=[shape=rectangle,draw,minimum size=\agrelsize]
\newcommand{\agcoordpary}{6}   %% y-coordinate of parent attributes
\newcommand{\agcoordchldy}{0}  %% y-coordinate of child attributes
\newcommand{\agcoordlocy}{3}   %% y-coordinate of local attributes
\newcommand{\agbeginprod}[5]{} %% you could begin a picture here
\newcommand{\agendprod}[5]{}   %% you could end a picture here
\newcommand{\agparent}[5]{ \draw (#4 \agrelsize,\agcoordpary \agrelsize) node [style=agnontermstyle] (#3) {}; \draw (#3.north) node [anchor=south] {\footnotesize{ #2 }}; \draw (#3.south) node [anchor=north] {\footnotesize{ #5 }}; }
\newcommand{\agparattr}[6]{ \draw (#4 \agrelsize,\agcoordpary \agrelsize) node [style=#6] (#3) {}; \draw (#3.south) node [anchor=south] {\tiny{#2}}; }
\newcommand{\agchldattr}[6]{ \draw (#4 \agrelsize,\agcoordchldy \agrelsize) node [style=#6] (#3) {}; \draw (#3.north) node [anchor=north] {\tiny{#2}}; }
\newcommand{\agchldnonterm}[5]{ \draw (#4 \agrelsize,\agcoordchldy \agrelsize) node [style=agnontermstyle] (#3) {}; \draw (#3.south) node [anchor=north] {\footnotesize{ #2:#5 }}; }
\newcommand{\agchldterm}[5]{ \draw (#4 \agrelsize,\agcoordchldy \agrelsize) node [style=agchldstyle] (#3) {}; \draw (#3.south) node [anchor=north] {\footnotesize{ #2 }}; }
\newcommand{\agsibline}[3]{ \draw (#2.east) -- (#3.west); }
\newcommand{\agdeplineupdown}[4]{ \draw[->,style=agdepstyle] (#2.north) .. controls +(0,1) and +(0,-1) .. (#3.south); }
\newcommand{\agdeplineupup}[5]{ \draw[->,style=agdepstyle] (#2.north) .. controls +(0,#5) and +(0,#5) .. (#3.north); }
\newcommand{\agidplineup}[5]{ \draw[->,style=agidpstyle] (#2.north) .. controls +(0,#5) and +(0,#5) .. (#3.north); }
\newcommand{\agidplineupdown}[5]{ \draw[->,style=agidpstyle] (#2.north) .. controls +(0,#5) and +(0,-#5) .. (#3.south); }
\newcommand{\agdeplineupupbold}[4]{ \draw[->,style=agdepstyle,line width=2pt] (#2.north) .. controls +(0,1) and +(0,1) .. (#3.north); }
\newcommand{\agdeplinedowndown}[5]{ \draw[->,style=agdepstyle] (#2.south) .. controls +(0,-#5) and +(0,-#5) .. (#3.south); }
\newcommand{\agidplinedown}[5]{ \draw[->,style=agidpstyle] (#2.south) .. controls +(0,-#5) and +(0,-#5) .. (#3.south); }
\newcommand{\agdeplinedowndownbold}[4]{ \draw[->,style=agdepstyle,line width=2pt] (#2.south) .. controls +(0,-1) and +(0,-1) .. (#3.south); }
\newcommand{\agdeplinedowndowndashed}[5]{ \draw[->,style=agidpstyle, dashed] (#2.south) .. controls +(0,-#5) and +(0,-#5) .. (#3.south); }
\newcommand{\agdeplinedownup}[4]{ \draw[->,style=agdepstyle] (#2.south) .. controls +(0,-1) and +(0,1) .. (#3.north); }
\newcommand{\agpicture}[1]{ \begin{tikzpicture}\input{#1.tikz}\end{tikzpicture} }

\newcommand{\mypic}[2]{
 \begin{figure}[ht]
  \centering
    \agpicture{diagrams/#1}
    \caption{#2}
  \label{fig:#1}
 \end{figure}
}

\newcommand{\mypicstar}[2]{
 \begin{figure*}[t]
  \centering
    \agpicture{diagrams/#1}
    \caption{#2}
  \label{fig:#1}
 \end{figure*}
}

\newcommand{\mydoublepic}[6]{
 \begin{figure*}[t]
  \centering
    \begin{minipage}[a]{#3\textwidth}
        \agpicture{diagrams/#2}
    \end{minipage}
    \begin{minipage}[a]{#5\textwidth}
        \agpicture{diagrams/#4}
    \end{minipage}
    \caption{#6}
  \label{fig:#1}
 \end{figure*}
}

\newcounter{algoctr}
\setcounter{algoctr}{-1}
\newcounter{cfctr}
\setcounter{cfctr}{-1}
\newcounter{defctr}

